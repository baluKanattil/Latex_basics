\documentclass[11pt]{article}	

\begin{document}

\section*{Tutorial 2}
Common math notations:

superscripts:
$$2x^3$$ 
$$2x^{34}$$
$$2x^{3x+4} $$
$$sx^{3x^4+5}$$

subscripts:
$$x_{12}$$
$${x_1}_2$$

greekletters:
$$\pi$$
$$\alpha$$
$$ A= \pi r^2$$

trig functions:
$$y=\sin(x)$$

log functions:
$$\log_5(x)$$
$$\ln(x)$$

roots:
$$\sqrt[2]{4}$$
$$\sqrt{x^2+y^2}$$

fractions:

About $\displaystyle{\frac{2}{3}}$ of glass is full
$$\frac{x^2}{\sqrt{x+1}}$$
$$\frac{1}{1+\frac{1}{x}}$$
$$\sqrt{\frac{x}{x+1}}$$

\section*{Explanation}
Use backslash to use any technique other than text.
Math mode italizes variables. So better use math mode for sin and cos.
Curly braces are used for grouping. No parantheses.
Simple frac makes to small to fit in inline. Use display style
In the case of fractions, just use frac and curly braces as needed.
\end{document}